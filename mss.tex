\documentclass[twoside]{article}

\usepackage{ustj}

\addbibresource{mss.bib}

\newcommand{\authorname}{N. E. Davis}
\newcommand{\authorpatp}{\patp{lagrev-nocfep}}
\newcommand{\affiliation}{Urbit Foundation}

%  Make first page footer:
\fancypagestyle{firststyle}{%
\fancyhf{}% Clear header/footer
\fancyhead{}
\fancyfoot[L]{{\footnotesize
              %% We toggle between these:
              Manuscript submitted for review.\\
              % {\it Urbit Systems Technical Journal} I:1 (2024):  1–3. \\
              ~ \\
              Address author correspondence to \authorpatp.
              }}
}
%  Arrange subsequent pages:
\fancyhf{}
\fancyhead[LE]{{\urbitfont Urbit Systems Technical Journal}}
\fancyhead[RO]{Subsecond Intervals}
\fancyfoot[LE,RO]{\thepage}

%%MANUSCRIPT
\title{A Note on Subsecond Base-2 Time Intervals}
\author{\authorname~\authorpatp \\ \affiliation}
\date{}

\begin{document}

\maketitle
\thispagestyle{firststyle}

\begin{abstract}
Urbit time provides for the representation of subsecond values down to $2^{-64}$ s.  Such a base-two numbering system prompts the consideration of prefixes for subsecond intervals that are analogous to the SI prefixes for positive powers of two, as used in memory sizes.  We propose a set of prefixes for subsecond intervals that are similar to the SI prefixes for negative powers of ten, and we suggest that these prefixes be used in Urbit time representations.  We also discuss the relative error in these prefixes compared to the SI prefixes for negative powers of ten.
\end{abstract}

% We will adjust page numbering in final editing.
\pagenumbering{arabic}
\setcounter{page}{1}

Urbit time is conventionally a 128-bit value,\footnote{Time can actually be an arbitrarily sized atom, but $2^{64} \,\textrm{s} \approx 58 \times 10^{9} \,\textrm{a}$; most discussions of noncosmological time will not involve such large intervals.} with the lower 64 bits denoting fractions of a second and the upper 64 bits denoting multiples of a second.  (That is, 1 s = \lstinline[style=inlinecode]{0x1.0000.0000.0000.0000}).

By expanding the atomic form of time (\lstinline[style=inlinecode]{@dr}) to the tuple form, we can clearly see how each component of the time is represented:

\begin{lstlisting}[style=listingcode]
> `@da`now
~2024.8.14..14.48.13..e5ea

> (yore `@da`now)
[[a=%.y y=2.024] m=8 t=[d=14 h=14 m=48 s=13 f=~[0xe5ea]]]
\end{lstlisting}

The last component of the tuple is a list of subsecond time intervals; while \lstinline[style=inlinecode]{now} does not feign to provide more accuracy than $2^{-16} \,\textrm{s}$, the underlying representation can provide for subsecond intervals down to $2^{-64} \,\textrm{s}$ by including more values in the \texttt{f} list.

Decimal time does not cleanly map to this representation, and the resulting hexadecimal values are largely opaque to human interpretation:

\begin{itemize}
  \item $1 \,\textrm{ms} = 10^{-3} \,\textrm{s} = \mathtt{0x41.8937.4bc6.a7ef}$
  \item $1 \,\textrm{\mu s} = 10^{-6} \,\textrm{s} = \mathtt{0x10c6.f7a0.b5ed}$
  \item $1 \,\textrm{ns} = 10^{-9} \,\textrm{s} = \mathtt{0x4.4b82.fa09}$
  \item $1 \,\textrm{ps} = 10^{-12} \,\textrm{s} = \mathtt{0x119.7998}$
  \item $1 \,\textrm{fs} = 10^{-15} \,\textrm{s} = \mathtt{0x480e}$
  \item $1 \,\textrm{as} = 10^{-18} \,\textrm{s} = \mathtt{0x12}$
\end{itemize}

\noindent
Smaller intervals of a second cannot be represented in native Urbit time relative time \lstinline[style=inlinecode]{@dr}.

Urbit time can therefore be more exactly expressed in terms of binary fractions of a second rather than decimal fractions of a second.  However, while there are established expressions for powers of two that are close to positive powers of ten \citep{WikipediaByte}, there are no expressions for powers of two close to negative powers of ten.  We would like to introduce such a scheme for convenience in Urbit time.
% WikipediaByte https://en.wikipedia.org/wiki/Byte#Multiple-byte_units

\begin{itemize}
  \item $2^{-10} \,\textrm{s} = \mathtt{0x40.0000.0000.0000}$
  \item $2^{-20} \,\textrm{s} = \mathtt{0x10.0000.0000.0000}$
  \item $2^{-30} \,\textrm{s} = \mathtt{0x4.0000.0000}$
  \item $2^{-40} \,\textrm{s} = \mathtt{0x100.0000}$
  \item $2^{-50} \,\textrm{s} = \mathtt{0x4000}$
  \item $2^{-60} \,\textrm{s} = \mathtt{0x10}$
\end{itemize}

What prefixes should we provide?  The standard names for the positive values of ten are:

\begin{itemize}
  \item $2^{10} = \textrm{kibi} \approx 10^{3}$
  \item $2^{20} = \textrm{mebi} \approx 10^{6}$
  \item $2^{30} = \textrm{gibi} \approx 10^{9}$
  \item $2^{40} = \textrm{tebi} \approx 10^{12}$
  \item $2^{50} = \textrm{pebi} \approx 10^{15}$
  \item $2^{60} = \textrm{exbi} \approx 10^{18}$
\end{itemize}

This suggests that we should prefer prefixes that are similar to the standard names for the negative values of two but with a different last syllable.  Following a suggestion by \patp{rovyns-ricfer} for “-ki-”, we propose the following prefixes:

\begin{tabular}{llll}
  \textbf{Binary Value} & \textbf{Prefix} & \textbf{Symbol} & \textbf{Decimal Approximation} \\
  $2^{-10}$ & miki & mi & $10^{-3}$ \\
  $2^{-20}$ & muki & ui & $10^{-6}$ \\
  $2^{-30}$ & naki & ni & $10^{-9}$ \\
  $2^{-40}$ & piki & pi & $10^{-12}$ \\
  $2^{-50}$ & feki & fi & $10^{-15}$ \\
  $2^{-60}$ & akki & ai & $10^{-18}$ \\
\end{tabular}

As occurs with memory values denominated in kibi, mebi, gibi, etc., it will be important to keep in mind the relative error in such time expressions.  We can compare these binary magnitudes roughly to decimal magnitudes and calculate the resulting error, much as we do with the SI prefixes for positive values of ten and two:

\begin{tabular}{llll}
  \textbf{Binary Value} & \textbf{Prefix} & \textbf{Decimal Approximation} & \textbf{Error} \\
  $2^{-10}$ & miki & $10^{-3}$ & 2.4\% \\
  $2^{-20}$ & muki & $10^{-6}$ & 4.9\% \\
  $2^{-30}$ & naki & $10^{-9}$ & 7.4\% \\
  $2^{-40}$ & piki & $10^{-12}$ & 9.9\% \\
  $2^{-50}$ & feki & $10^{-15}$ & 12.6\% \\
  $2^{-60}$ & akki & $10^{-18}$ & 12.5\% (truncation error) \\
\end{tabular}

We trust that the existing ability to express subsecond intervals in Urbit time will continue to prove useful, and furthermore that the proposed prefixes will be helpful in promoting legibility and accuracy.

\selectlanguage{USenglish}
\printbibliography
\end{document}
