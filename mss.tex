\documentclass[twoside]{article}

\usepackage{ustj}
\usepackage{tabularx}

\addbibresource{mss.bib}

\newcommand{\authorname}{N. E. Davis}
\newcommand{\authorpatp}{\patp{lagrev-nocfep}}
\newcommand{\affiliation}{Urbit Foundation}

%  Make first page footer:
\fancypagestyle{firststyle}{%
\fancyhf{}% Clear header/footer
\fancyhead{}
\fancyfoot[L]{{\footnotesize
              %% We toggle between these:
              Manuscript submitted for review.\\
              % {\it Urbit Systems Technical Journal} I:1 (2024):  1–3. \\
              ~ \\
              Address author correspondence to \authorpatp.
              }}
}
%  Arrange subsequent pages:
\fancyhf{}
\fancyhead[LE]{{\urbitfont Urbit Systems Technical Journal}}
\fancyhead[RO]{Subsecond Intervals}
\fancyfoot[LE,RO]{\thepage}

%%MANUSCRIPT
\title{A Note on Subsecond \\ Base-2 Time Intervals}
\author{\authorname~\authorpatp \\ \affiliation}
\date{}

\begin{document}

\maketitle
\thispagestyle{firststyle}

\begin{abstract}
Urbit time provides for the representation of subsecond values down to $2^{-64}$ s.  Such a base-two numbering system prompts the consideration of prefixes for subsecond intervals that are analogous to the SI prefixes for positive powers of two, as used in memory sizes.  We propose a set of prefixes for subsecond intervals that are similar to the SI prefixes for negative powers of ten, and we suggest that these prefixes be used in Urbit time representations.  We also discuss the relative error in these prefixes compared to the SI prefixes for negative powers of ten.
\end{abstract}

% We will adjust page numbering in final editing.
\pagenumbering{arabic}
\setcounter{page}{1}

Urbit time is conventionally a 128-bit atom,\footnote{Time can actually be an arbitrarily sized atom, but $2^{64} \,\textrm{s} \approx 58 \times 10^{9} \,\textrm{a}$; most discussions of noncosmological time will not involve such large intervals.} with the lower 64 bits denoting fractions of a second and the upper 64 bits denoting multiples of a second.  (That is, 1 s = \lstinline[style=inlinecode]{(bex 64)}.)

By expanding the atomic form of time (\lstinline[style=inlinecode]{@dr}) to the tuple form, we can clearly see how each component of the time is represented:

\begin{lstlisting}[style=listingcode]
  > (yell now)  :: Treating "now" as an interval.
  [d=106.751.991.823.894 h=15 m=19 s=22 f=~[0xd43]]
\end{lstlisting}

The last component of the tuple is a list of subsecond time intervals; while \lstinline[style=inlinecode]{now} does not feign to provide more accuracy than $2^{-16} \,\textrm{s}$, the underlying representation can provide for subsecond intervals down to $2^{-64} \,\textrm{s}$ by including more values in the \texttt{f} list.

\begin{lstlisting}[style=listingcode]
  > (yell `@dr`0x1)
  [d=0 h=0 m=0 s=0 f=~[0x0 0x0 0x0 0x1]]
\end{lstlisting}

Subsecond decimal time does not cleanly map to this representation, and the resulting hexadecimal values are largely opaque to human interpretation:

\begin{tabularx}{\textwidth}{lll}
  \textbf{Interval} & \textbf{Decimal Value} & \textbf{Hexadecimal Value} \\
  $1 \,\textrm{ms}$ & $10^{-3} \,\textrm{s}$ & $\mathtt{0x41.8937.4bc6.a7ef}$ \\
  $1 \,\textrm{µs}$ & $10^{-6} \,\textrm{s}$ & $\mathtt{0x10c6.f7a0.b5ed}$ \\
  $1 \,\textrm{ns}$ & $10^{-9} \,\textrm{s}$ & $\mathtt{0x4.4b82.fa09}$ \\
  $1 \,\textrm{ps}$ & $10^{-12} \,\textrm{s}$ & $\mathtt{0x119.7998}$ \\
  $1 \,\textrm{fs}$ & $10^{-15} \,\textrm{s}$ & $\mathtt{0x480e}$ \\
  $1 \,\textrm{as}$ & $10^{-18} \,\textrm{s}$ & $\mathtt{0x12}$ \\
\end{tabularx}

\noindent
Smaller intervals of a second than $2^{-64}$ cannot be represented in native Urbit time relative time \lstinline[style=inlinecode]{@dr}.

Urbit time can therefore be more exactly expressed in terms of binary fractions of a second rather than decimal fractions of a second.  However, while there are established expressions for powers of two that are close to positive powers of ten, there are no expressions for powers of two close to negative powers of ten.  We would like to introduce such a scheme for convenience in Urbit time.

\begin{tabularx}{\textwidth}{ll}
  \textbf{Binary Value} & \textbf{Hexadecimal Value} \\
  $2^{-10} \,\textrm{s}$ & $\mathtt{0x40.0000.0000.0000}$ \\
  $2^{-20} \,\textrm{s}$ & $\mathtt{0x10.0000.0000.0000}$ \\
  $2^{-30} \,\textrm{s}$ & $\mathtt{0x4.0000.0000}$ \\
  $2^{-40} \,\textrm{s}$ & $\mathtt{0x100.0000}$ \\
  $2^{-50} \,\textrm{s}$ & $\mathtt{0x4000}$ \\
  $2^{-60} \,\textrm{s}$ & $\mathtt{0x10}$ \\
\end{tabularx}

\noindent
What prefixes should we provide?  The standard names for the positive values of ten are:

\begin{tabularx}{\textwidth}{lll}
  \textbf{Binary Value} & \textbf{Prefix} & \textbf{Approximation} \\
  $2^{10}$ & kibi & $10^{3}$ \\
  $2^{20}$ & mebi & $10^{6}$ \\
  $2^{30}$ & gibi & $10^{9}$ \\
  $2^{40}$ & tebi & $10^{12}$ \\
  $2^{50}$ & pebi & $10^{15}$ \\
  $2^{60}$ & exbi & $10^{18}$ \\
\end{tabularx}

This suggests that we should prefer prefixes that are similar to the standard names for the negative values of two but with a different last syllable.  Following a suggestion by \patp{rovyns-ricfer} for “-ki-”, we propose the following prefixes:

\begin{tabularx}{\textwidth}{llll}
  \textbf{Binary Value} & \textbf{Prefix} & \textbf{Symbol} & \textbf{Approximation} \\
  $2^{-10}$ & miki & mi & $10^{-3}$ \\
  $2^{-20}$ & muki & µi or ui & $10^{-6}$ \\
  $2^{-30}$ & naki & ni & $10^{-9}$ \\
  $2^{-40}$ & piki & pi & $10^{-12}$ \\
  $2^{-50}$ & feki & fi & $10^{-15}$ \\
  $2^{-60}$ & akki & ai & $10^{-18}$ \\
\end{tabularx}

As occurs with memory values denominated in kibi, mebi, gibi, etc., it will be important to keep in mind the relative error in such time expressions.  We can compare these binary magnitudes roughly to decimal magnitudes and calculate the resulting error, much as we do with the SI prefixes for positive values of ten and two:

\begin{tabularx}{\textwidth}{llll}
  \textbf{Binary Value} & \textbf{Prefix} & \textbf{Approximation} & \textbf{Error} \\
  $2^{-10}$ & miki & $10^{-3}$ & 2.4\% \\
  $2^{-20}$ & muki & $10^{-6}$ & 4.9\% \\
  $2^{-30}$ & naki & $10^{-9}$ & 7.4\% \\
  $2^{-40}$ & piki & $10^{-12}$ & 9.9\% \\
  $2^{-50}$ & feki & $10^{-15}$ & 12.6\% \\
  $2^{-60}$ & akki & $10^{-18}$ & 12.5\%\footnote{This is due to truncation error; it ``should'' be 15\%.} \\
\end{tabularx}

We trust that the current ability to express subsecond intervals in Urbit time will continue to prove useful, and furthermore that the proposed prefixes will promote legibility and accuracy. \tombstone{}

\selectlanguage{USenglish}
\printbibliography
\end{document}
